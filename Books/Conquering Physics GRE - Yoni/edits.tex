\documentclass{article}
\usepackage[utf8]{inputenc}
\usepackage{amsmath,amsthm,amssymb}
\usepackage{enumitem}
\usepackage{mathrsfs}

\title{Corrections to "Conquering the Physics GRE"}
\author{Sai Krishna Deep}
\date{24th October 2015}

\begin{document}

\maketitle

Your book has been of IMMENSE help in all aspects. I'm extremely thankful to this book.Few forumulas literally just poped out from your book! Also, the application of you're formulas and the emphasis on the usage of certain formulas like the "reduced mass" in the "Bohr's Radius" was spot on. Below are a changes that I felt was needed. The following are the ones that I thought were eseential to be added. A few of them even appeared on my exam which is why I'm sending these corrections in the hope of helping future GRE apirants! \\

\section{Classical Mechanics}
\begin{enumerate}
\item \textbf{Pg 21} - A small note on perpendicular axis theorem $I_x+I_y = I_z$ could probably help as well.
\item \textbf{Pg 30} - Conventions of Euler's Angles, $\theta,\phi,\gamma$ might help.
\item \textbf{Pg 31} - Example of \textit{Reduced Mass} systems might help in general problems.
\end{enumerate}
\section{Electromagnetism}
\begin{enumerate}
\item \textbf{Pg 60} - Basic trivial vector calculus identies such as $\nabla \times (\nabla\cdot A) =0$ and other such combinations of $\nabla,\times,\cdot$ will certainly help in manipulating maxwell's equations. $\nabla^2$.
\item \textbf{Pg 60} - A small note on the fact that \textit{potential is independent of the path} might also help when we are looking at conserved forces.
\item \textbf{Pg 60} - Energry stored = -ve of work done by conservative force.
\item \textbf{Pg 66} - A note on volume element $r^2\sin \theta dr d\theta d\phi$ and surface element $r^2 d\phi d\theta$ in polar co-ordinates.
\item \textbf{Pg 73} - Work done by battery is $qV = CV^2$ which might help in solving energry relaed problems in circuits.
\item \textbf{Pg 90} - Impication of di-electric constant $k$. By what factor the $Q$ and energry charge. Also, a note on the fact that keeping $Q$ fixed during the insertion of dielectric implies $V$ to be variable and vice versa.
\end{enumerate}
\section{Optics}
\begin{enumerate}
\item \textbf{Pg118} - General len's maker formula as a result of combination of lens formula at curved surfaces.
\item \textbf{Pg 130} - The change of phase of Electric and Magnetic fields due to a reflecting / refracting interface. The forumla for transmission and reflection amplitudes
\end{enumerate}
\section{Quantum Mechanics}
\begin{enumerate}
\item \textbf{Pg 157} - You accidently wrte $c$ to be the eigenvalue instead of $\lambda$.
\item \textbf{Pg 159} - You defined what the left multiplication of the operator $A$ is. But it is unclear what it means to say $<A^\dagger a \mid$ when you have just mentioned what $\mid A^\dagger a>$ means.
\item \textbf{Pg 164} - The expression for a guassian wave packet is given in $J.J.Sakurai$ in quite a useful format. Probably you wrote write a small note on that too.
\item \textbf{Pg 168} - Classical implication of ground state $\mid 0 >$ and other general $\mid n>$
\item \textbf{Pg 172} - A small note on \textit{Scattering} and \textit{Bound} states might help.
\end{enumerate}
\section{Statistical Mechanics}
\begin{enumerate}
\item \textbf{Pg 236} - Write a section with just basic quantites with their dimensions/ a small formula to quickly get its units like $\hbar$. Also the units of a few less common qunaties whose related equations may not be well known, like $candela, decibel, etc.$ could also be asked as trivia.
\item \textbf{Pg 243} - A few basic values in terms of $eV$ will help like $\hbar c, m_ec^2,931 MeV, 1230 MeV$ etc. (In a calculation I was required to know $m_ec^2$ in $eV$, unfortunately I had to skip this due to lack of time.)
\item \textbf{Pg 245} - Finding the binding energy per nucleon by mass defect formula $\Delta m c^2$ was asked in a few past GRE papers. Also, there is a popular graph of few elemnts with their binding energies per nucleon. This might help.
\item \textbf{Pg 245} - A few common notations such as $A$ - massnumber,$Z$- atomic number might help understanding the decay of elements such as $_7^{14}N \rightarrow _6^12 C$ interms of number of protons, electron etc.
\item \textbf{Pg 245} - Spectroscopic notation has been asked in my exam and aslo the revious exams. Inclusion of this will be of immense help.
\item \textbf{Pg 245} - I was asked a question of the basic setup of \textit{Stern-Gerlach} experiment. Inclusion of a basic idea of this in this chapter as an intro as in any freshman course will definitely be benefical.
\end{enumerate}
\section{Special Relativity}
\begin{enumerate}
\item \textbf{Pg 227} - VIBGYOR - as an arrangement from Highest to Lowest frquency will help in relating different colors frequncies,wavelength and energies.
\item \textbf{Pg 227} - A small reminder of \textit{componendo-dividendo} will help in SPR.
\end{enumerate}
\section{Questions based on previous papers}
\begin{enumerate}
\item \textbf{Oscilliscope} - Basic questions invlving oscilloscpe has been asked consistently on every GRE. It will be useful to include this in one of your sections with a bit more detail.
\item \textbf{Carnot Engine} - I havn't seen a basic understaing of this formula apart from heat reservoirs. And the word itself (I had a question on efficincy of a carnot cycle this time like in all other exams.)
\item \textbf{Decibel} - Intensity of sound was asked in my paper this time. It might be useful to include this. A small formula perhaps.
\item \textbf{Binding Energy} - I dont recall any special emphasis on Binding Energy in your material. I was asked this time. Also, it appeared on the 2008 paper. A section for this might be of some use.
\end{enumerate}
\end{document}
